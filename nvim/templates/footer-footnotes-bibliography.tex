\documentclass{article}
% \documentclass[twocolumn]{article}

% \usepackage{pdfsync} % ggf für Synchronization mit skiim, aber ggf wegen Serverlösung gar nicht mehr notwendig


\input{/Users/g/Library/Mobile Documents/com~apple~CloudDocs/#Docs iCloud/R Statistik, icloud/Latex, icloud/Latex Projekte/@Vorlagen-Latex/preamble-footnotes-bibliographie-neu.tex}


%-------------------------- TEXT FUßZEILE ----------------------------

\rfoot{\hfill ____, G. Haarmeyer, Stand: \today, S.  \thepage} % Zentrale Fußzeile mit Datum und Seitenzahl rechts
\lfoot{\hyperlink{toc}{Zurück zum Inhaltsverzeichnis}} % Linke Seite der Fußzeile
%========================== 

%========================== TITEL ============================
\usepackage{titling}
% Define a new command for the subtitle
\newcommand{\subtitle}[1]{
    \posttitle{
        \par\large#1\end{center}
    }
}

\title{____}
\subtitle{- nur für den persönlichen Gebrauch -}

\date{} % wenn leer kein Datum

\begin{document}
%\maketitle % Erstellt das Titelblatt
\thispagestyle{fancy} % Ermöglicht die Fußzeile auf der Titelseite
\hypertarget{toc}{}

% Hyperlinks schwarz für das Inhaltsverzeichnis
\hypersetup{linkcolor=black}
% Anpassen der Überschriftsabstände

%-------------------------- INHALTSVERZEICHNIS ----------------------------  
\tableofcontents % Erzeugt das Inhaltsverzeichnis

% Hyperlinks dunkelrot für den Rest des Dokuments
\hypersetup{
  linkcolor = red!70!black,
  citecolor = black, % Farbe für Zitate
  urlcolor = blue,    % Farbe für URL-Links
% Änderung nur der Fußnotenlinkfarbe:
  footnotecolor = black
}


%%%%%%%%%%%%%%%%%%%%%%%%%%%%%%%%%%%%%%%%%%%%%%%%%%%%%

                    % BEGINN DOKUMENT

%%%%%%%%%%%%%%%%%%%%%%%%%%%%%%%%%%%%%%%%%%%%%%%%%%%%%

\section*{Kernpunkte} % k STernchen nach section macht dass die sction nicht nummeriert wird und auch nicht im Inhaltsverzeichnis erscheint
\begin{enumerate}
    \item Key Point 1
    \item Key Point 2
    \item Key Point 3
\end{enumerate}


\begin{multicols}{2}



    


\end{multicols}
\vspace{20 mm} % Fügt einen vertikalen Abstand ein
% Drucken des Literaturverzeichnisses
\printbibliography

\end{document}
