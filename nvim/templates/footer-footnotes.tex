\documentclass{article}

\input{/Users/g/Library/Mobile Documents/com~apple~CloudDocs/#Docs iCloud/R Statistik, icloud/Latex, icloud/Latex Projekte/@Vorlagen-Latex/preamble-footnotes-bibliographie-neu.tex}


\usepackage{titlesec} % kann z.B. subsubsubsubsections erstellen, kann Schriftgröße, Schriftart und Farbe von Titeln anpassen.

\titlespacing*{\section}{0pt}{2.5ex plus 1ex minus .2ex}{1.3ex plus .2ex}


%========================== KOPF - UND FUßZEILE ============================
\usepackage{fancyhdr} % für Fußzeilen
\renewcommand{\headrulewidth}{0pt} % Entfernt die Linie unter der Kopfzeile
%
\renewcommand{\footrulewidth}{0.4} % Fügt eine Linie oberhalb der Fußzeile auf jeder Seite ein


\pagestyle{fancy} % definiert den style "fancy"
\fancyhf{} % Löscht die aktuellen Einstellungen für Kopf- und Fußzeile

\rfoot{\hfill ____, G. Haarmeyer, Stand: \today, S.  \thepage} % Zentrale Fußzeile mit Datum und Seitenzahl rechts
\lfoot{\hyperlink{toc}{Zurück zum Inhaltsverzeichnis}} % Linke Seite der Fußzeile


%========================== TITEL ============================

\title{____
	\vspace{-2em} % Verringert Abstand zwischen Titel und der darauf folgenden Zeile,"em" bezieht sich auf den Abstand des Buchstaben "M" (Höhe ist abh. von Schriftgröße
	}
\date{} % wenn leer kein Datum


\begin{document}
\maketitle % Erstellt das Titelblatt
\thispagestyle{fancy} % Ermöglicht die Fußzeile auf der Titelseite
\hypertarget{toc}{}

% Hyperlinks schwarz für das Inhaltsverzeichnis
\hypersetup{linkcolor=black}
% Anpassen der Überschriftsabstände

%-------------------------- INHALTSVERZEICHNIS ----------------------------  
\tableofcontents % Erzeugt das Inhaltsverzeichnis

% Hyperlinks dunkelrot für den Rest des Dokuments
\hypersetup{linkcolor = red!70!black}

%%%%%%%%%%%%%%%%%%%%%%%%%%%%%%%%%%%%%%%%%%%%%%%%%%%%%

                    % BEGINN DOKUMENT

%%%%%%%%%%%%%%%%%%%%%%%%%%%%%%%%%%%%%%%%%%%%%%%%%%%%%


\vspace{20 mm} % Fügt einen vertikalen Abstand ein
% Drucken des Literaturverzeichnisses


\end{document}
