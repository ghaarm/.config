\documentclass{article}


%========================== GRAPHIKEN & PDFS ============================

\usepackage{graphicx} % Required for inserting images
\usepackage[space]{grffile} % ermöglicht Leerzeichen in den Dateinamen

\usepackage{xcolor}

\usepackage{pdfpages} %ermöglich das Einfügen von PDFs z.B mit includeodf

\usepackage{pdflscape} % macht landscape = Querformat

\usepackage{float} % kann Graphiken, Tabellen genau an einer bestimmten Stelle einfügen

%========================== "GLOBALE" Formatierung ============================
\usepackage[utf8]{inputenc} % Erlaubt die direkte Eingabe von UTF-8-kodierten Zeichen
\usepackage[T1]{fontenc}    % Verwendet die T1-Schriftcodierung für die Ausgabe

\usepackage[ngerman]{babel} % macht deutsche Lokalisierung z.B. Datumformate

\usepackage{datetime2} % macht z.B. aktuelles Datum

\usepackage{amsmath} % für mathematische Zeichen
%\usepackage{amssymb} % für mathematische Zeichen


%========================== TEXTKÖRPER ============================
\usepackage[textwidth=18cm]{geometry} % kann die Textbreite auf dem Paper ändern

\geometry{ 
a4paper, % Setzt das Papierformat auf A4
  textwidth = 18 cm, % Textbreite
  textheight = 26 cm % Texthöhe
}
%-------------------------- TEXT FORMATIERUNG ----------------------------

\usepackage{parskip} % macht einen vertikalen Einzug anstatt eines Absatzes

\usepackage[normalem]{ulem} % Die Option 'normalem' behält die übliche Betonung bei (\emph{} wird nicht unterstrichen

\usepackage{csquotes} % macht Anführungnszeichenn

\newcommand{\ignore}[1]{} % um einzelne Wörter auszublenden

\usepackage{multicol}
%========================== HYPERLINKS ============================

\usepackage{hyperref}
\hypersetup{
			colorlinks = true,	
			linkcolor = red!70!black,
			citecolor = black,
			urlcolor = blue
			}


\usepackage{titlesec} % kann z.B. subsubsubsubsections erstellen, kann Schriftgröße, Schriftart und Farbe von Titeln anpassen.

\titlespacing*{\section}{0pt}{2.5ex plus 1ex minus .2ex}{1.3ex plus .2ex}

%========================== KOPF - UND FUßZEILE ============================
\usepackage{fancyhdr} % für Fußzeilen
\renewcommand{\headrulewidth}{0pt} % Entfernt die Linie unter der Kopfzeile
%
\renewcommand{\footrulewidth}{0.4} % Fügt eine Linie oberhalb der Fußzeile auf jeder Seite ein

\pagestyle{fancy} % definiert den style "fancy"
\fancyhf{} % Löscht die aktuellen Einstellungen für Kopf- und Fußzeile

\rfoot{\hfill replace-date, S.  \thepage\ von 2} % Zentrale Fußzeile mit Datum und Seitenzahl rechts
\lfoot{Übergabeprotokoll \enquote{An der Krücke 36} 1. OG links, G. Haarmeyer, Stand: 07/2024} % Linke Seite der Fußzeile
%
%========================== TITEL ============================
\usepackage{titling}
% Define a new command for the subtitle
\newcommand{\subtitle}[1]{
    \posttitle{
        \par\large#1\end{center}
    }
}

\title{Übergabeprotokoll}
\subtitle{- bei Einzug -}

\date{replace-date} % wenn leer kein Datum

\begin{document}
\maketitle
\thispagestyle{fancy}
%%%%%%%%%%%%%%%%%%%%%%%%%%%%%%%%%%%%%%%%%%%%%%%%%%%%%

                    % BEGINN DOKUMENT

%%%%%%%%%%%%%%%%%%%%%%%%%%%%%%%%%%%%%%%%%%%%%%%%%%%%%

Zwischen

Golo-Sung Haarmeyer

Blütenstraße 22

90480 Nürnberg

\hfill - folgend Vermieter genannt -

und

Name

geboren am: 

derzeitiger Wohnsitz: 

\hfill - folgend Mieter:in genannt - 

\vspace{1cm}

\textbf{Objekt: Mietwohnung 1. OG links in \enquote{An der Krücke 36}}

Hiermit bestätige ich, dass mir die Mieträume, entsprechend § 8 des Mietvertrages vom replace-vertrag, übergeben wurden.

\section{Zählerstände}%
  \label{sec:Zählerstände}
  
    \begin{tabular}{|l|r|r|}
      \hline
      Ort & Zählerstände & Zählernummer \\
      \hline
      Küche (Wasser, m\textsuperscript{3}): & Zelle & 2 412 501 \\
      \hline
      Bad (Wasser, m\textsuperscript{3}): & Zelle & 2 412 499 \\
      \hline
      Waschkeller (Wasser, m\textsuperscript{3}, 3. Anschluss von links): & Zelle & 08 358 847 \\
      \hline
      Gas (m\textsuperscript{3}) & Zelle & 162 832 \\
      \hline
      Strom (kW/h) & Zelle & 457 024 \\
      \hline
    \end{tabular}

\section{Schlüssel}%
  \label{sec:Schlüssel}
  
Folgende Schlüssel wurden ausgehändigt:
    \newline
    \newline
    \begin{tabular}{|l|r|r|}
      \hline
      Beschreibung & Anzahl \\
      \hline
      Wohnungstür & 2 \\
      \hline
      Briefkasten & 2 \\
      \hline
      Haustür & 2 \\
      \hline
      Haushintertür & 2 \\
      \hline
      Schloss Keller & 2 \\
      \hline
    \end{tabular}


\section{Bemerkungen}%
  \label{sec:Bemerkungen}
  
    \begin{enumerate}
      \item 
      \item 
      \item
      \item 
    \end{enumerate}
\end{enumerate}


\vspace{3cm}

Bielefeld, replace-date

\rule{5cm}{0.3mm}

Ort, Datum

\vspace{1.5cm}

\noindent\begin{minipage}{0.5\textwidth}
Unterschrift

\vspace{2cm} % Vertikaler Abstand für die Unterschrift

\rule{5cm}{0.2mm} % Unterschriftslinie

Mieter:in
\end{minipage}
\hfill % Füllt den horizontalen Raum zwischen den minipages
\begin{minipage}{0.5\textwidth}
Unterschrift

\vspace{1.5cm} % Vertikaler Abstand für die Unterschrift

\rule{5cm}{0.2mm} % Unterschriftslinie

Vermieter
\end{minipage}

% Seitenzahl anpassen
% \newpage
% \begin{landscape}
%   \section{Anhang}%
%   \label{sec:Anhang}
%
%   \begin{multicols}{2}
%
%
%
%   \end{multicols}
% \end{landscape}
  
\end{document}

